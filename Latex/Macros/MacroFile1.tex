% This file contains macros that can be called up from connected TeX files
% It helps to summarise repeated code, e.g. figure insertion (see below).

% insert a centered figure with caption and description
% parameters 1:filename, 2:title, 3:description and label
\newcommand{\figuremacro}[3]{
	\begin{figure}[htbp]
		\centering
		\includegraphics[width=1\textwidth]{#1}
		\caption[#2]{\textbf{#2} - #3}
		\label{#1}
	\end{figure}
}

% insert a centered figure with caption and description AND WIDTH
% parameters 1:filename, 2:title, 3:description and label, 4: textwidth
% textwidth 1 means as text, 0.5 means half the width of the text
\newcommand{\figuremacroW}[4]{
	\begin{figure}[htbp]
		\centering
		\includegraphics[width=#4\textwidth]{#1}
		\caption[#2]{\textbf{#2} - #3}
		\label{#1}
	\end{figure}
}

% inserts a figure with wrapped around text; only suitable for NARROW figs
% o is for outside on a double paged document; others: l, r, i(inside)
% text and figure will each be half of the document width
% note: long captions often crash with adjacent content; take care
% in general: above 2 macro produce more reliable layout
\newcommand{\figuremacroN}[3]{
	\begin{wrapfigure}{o}{0.5\textwidth}
		\centering
		\includegraphics[width=0.48\textwidth]{#1}
		\caption[#2]{{\small\textbf{#2} - #3}}
		\label{#1}
	\end{wrapfigure}
}

% predefined commands by Harish
\newcommand{\PdfPsText}[2]{
  \ifpdf
     #1
  \else
     #2
  \fi
}

\newcommand{\IncludeGraphicsH}[3]{
  \PdfPsText{\includegraphics[height=#2]{#1}}{\includegraphics[bb = #3, height=#2]{#1}}
}

\newcommand{\IncludeGraphicsW}[3]{
  \PdfPsText{\includegraphics[width=#2]{#1}}{\includegraphics[bb = #3, width=#2]{#1}}
}

\newcommand{\InsertFig}[3]{
  \begin{figure}[!htbp]
    \begin{center}
      \leavevmode
      #1
      \caption{#2}
      \label{#3}
    \end{center}
  \end{figure}
}


%%% Local Variables: 
%%% mode: latex
%%% TeX-master: "~/Documents/LaTeX/CUEDThesisPSnPDF/thesis"
%%% End: 
\newcommand{\norm}[1]{\left\lVert#1\right\rVert}

\newcommand\numberthis{\addtocounter{equation}{1}\tag{\theequation}}