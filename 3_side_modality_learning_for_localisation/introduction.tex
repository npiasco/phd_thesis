One of the main challenges of \ac{vbl} remains the mapping of images acquired under changing conditions: cross-season images matching~\cite{Naseer2017a}, long-term localization~\cite{Toft2018}, day to night place recognition~\cite{Torii2015}, etc. Recent approaches use complementary information in order to address these visually challenging localization scenarios (geometric information through point cloud~\cite{Sattler2018,Schonberger2018} or depth maps~\cite{Christie2016}, semantic information~\cite{Ardeshir2014,Christie2016,Naseer2017a}). However geometric or semantic information are not always available or can be costly to obtain, especially in robotic or mobile applications when the sensor or the computational load on the system is limited.

In this paper, we propose a image descriptor capable of reproducing the underlying scene geometry from an monocular image, in order to deal with challenging outdoor large-scale image-based localization scenarios. We introduce dense geometric information as side training objective to make our new descriptor robust to visual changes that occur between images taken at different times. Once trained, our system can be used on monocular images only to construct a expressive descriptor for image retrieval. This kind of system design is also known as side information learning~\cite{Hoffman2016}, as it uses geometric and radiometric information only during the training step and pure radiometric data for the image localization. 

The paper is organized as follows. In section~\ref{sec:related_work}, we first revisit recent works related to our method, including:~state of the art image descriptors for large scale outdoor localization, method for localization in changing environment, side information learning approaches and depth from monocular for localization. In section~\ref{sec:method}, we describe in detail our new image descriptor trained with side depth information. In section~\ref{sec:impl_details} we give insight on our implementation and the dataset we used and we illustrate the effectiveness of the proposed method on six challenging scenarios in section~\ref{sec:experiments}. We discuss in section~\ref{sec:chall_loc} about the challenging night to day localization scenario and in section~\ref{sec:modality_ref} we present a variation of our method using dense object reflectance map instead of depth maps. Section~\ref{sec:conclusion} finally concludes the paper.

