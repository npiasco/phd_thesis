\section{Method}
\label{sec:method}

\begin{figure}
	\centering
	\includegraphics[width=\linewidth]{method/common_pipeline}	
	\caption[Pipeline of our pose refinement method]{\label{fig:pipeline} Pipeline of the proposed method. \textbf{a)} We retrieve initial pose of an image query using CBIR. \textbf{b)} We refine initial pose with a PnP algorithm where 2D to 3D matches are obtained through the reconstructed depth map of the reference image. \textcolor{purple}{Purple boxes} are deep features blocks used for dense images matching.}
\end{figure}


%\subsection{Method overview}
%\label{subsec:overview}

\paragraph{Workflow.}
Our method for fast image pose estimation is described in figure~\ref{fig:pipeline}. The camera pose is estimated following this two-step algorithm:
\begin{itemize}
	\item[\textbf{a)}] We obtain the initial pose of the query image by \ac{cbir} (section~\ref{subsec:image_indexing}).
	\item[\textbf{b)}] Initial pose is refined by finding dense correspondences between the query image and the best retrieved image (section~\ref{subsec:matching}). Meanwhile, we use a neural network to create the depth map related to the retrieved image candidate (section~\ref{subsec:depth_map}). We use correspondences between the 2D points of the query image and the 3D points projected from the depth map to compute the real pose of the query using Perspective-n-Point (PnP) algorithm (section \ref{subsec:pnlp}). We further denote our pose refinement method as Perspective-n-learned-Point (PnlP).
\end{itemize}

%Our method is fast and lightweight as it uses the same encoder/decoder neural architecture to compute the image descriptor, the correspondences between images and depth map associated to the retrieved image. It is also versatile as it required only radiometric data and poses of reference images during the online querying process, no heavy geometric scene representation as in~\citep{Taira2018}.

\paragraph{Notations.}
The aim of our method is to recover the camera pose $\mathbf{h_q}\in\mathbb{R}^{4\times 4}$, represented by a pose matrix in homogeneous coordinates, corresponding to an input RGB image $\mathrm{I_q}\in\mathbb{R}^{3\times H\times W}$. We know the matrix $\mathbf{K}\in\mathbb{R}^{3\times 3}$ of intrinsic parameters of the camera. We assume that we know the pose $\left\{ \mathbf{h_r}^{i} \right\}_{i=1,\ \ldots ,\ N}$ of a pool of $N$ references images $\left\{ \mathrm{I_{r}}^{i} \right\}_{i=1,\ \ldots ,\ N}$ of the scene where we want to localise the query. These poses can be obtained by SfM or by using external sensors. We denote as $\mathrm{E}$, respectively $\mathrm{D}$, a neural network encoder, respectively decoder.

\subsection{Image retrieval}
\label{subsec:image_indexing}

We cast the initial pose estimation task as a content-based image retrieval problem like in~\cite{Balntas2018}, since the reference data are augmented with 6 DoF pose information. In order to evaluate the similarity between the unknown pose query image $\mathrm{I_{q}}$ and the $N$ reference images $\left\{\mathrm{I_{r}}^{i} \right\}_{i=1,\ \ldots ,\ N}$, we need to use a discriminative image representation. Recent works have shown that deep features extracted from convolutional neural network offer better global image representations compared to hand-crafted features~\cite{Razavian2014a, Arandjelovic2017, Gordo2017, Radenovic2017}. We use a state-of-the art global image descriptor for place recognition, NetVLAD~\cite{Arandjelovic2017}, to describe the data by low-dimensional $L_2$ normalised vectors. The NetVLAD descriptor $\mathbf{f}$ is obtained by concatenating the dense feature from neural network encoder $\mathrm{E}$: $\mathbf{f} = \mathrm{NetVLAD}(\mathrm{E}(\mathrm{I}))$.

We first compute reference descriptors $\left\{ \mathbf{f}_{\mathrm{r}}^{\ i} \right\}_{i=1,\ \ldots ,\ N}$ from the reference images. Then we compare the query descriptor $\mathbf{f}_{\mathrm{q}}$ to the pre-computed descriptors by fast nearest neighbour indexing and retrieval:
\begin{equation}
	\left\{ \mathbf{\hat{f}}_{\mathrm{r}}^{\ j} \right\}_{j=1,\ \ldots ,\ K} = NN \left( \mathbf{f}_{\mathrm{q}}, \left\{ \mathbf{f}_{\mathrm{r}}^{\ i} \right\}_{i=1,\ \ldots ,\ N} \right),
\end{equation}
where $NN$ is the nearest neighbour matching function and $\mathbf{\hat{f}}_{\mathrm{r}}^{\ j}, j \in [1, K]$, the $K$ closest reference descriptors to the query descriptor. We use cosine similarity to evaluate the similarity between two descriptors and K-D tree as indexing structure. We consider poses $\mathbf{h}_\mathrm{r}^j, j \in [1, K],$ as candidate poses of the image $\mathrm{I_q}$.

\subsection{Dense correspondences}
\label{subsec:matching}

In order to refine the initial pose obtained by image retrieval, we compute correspondences between the query image and the closest retrieved image candidates. In~\citep{Taira2018, Noh2017, Widya2018}, authors use the dense features extracted by a convolutional neural network in order to compute correspondences between images. We follow the same idea and use the latent representation already computed by the neural network encoder $\mathrm{E}$ to compute correspondences between the query image and the $K$ retrieved candidates.

Local image descriptors are obtained from the latent image representation by concatenating the features at each position $\left( l, m \right)_{W_\mathrm{E},H_\mathrm{E}}$ ($W_\mathrm{E}$ and $H_\mathrm{E}$ are the spatial dimensions of the features map) along the depth of the features map~\citep{Taira2018, Widya2018}. We subsequently $L_2$-normalise the extracted descriptors before matching. We consider only consistence matches by rejecting correspondences that do not respect the bidirectional test (nearest descriptors of image 1 in image 2 have to be the same as nearest descriptors of image 2 to image 1).

\subsection{Depth from monocular image}
\label{subsec:depth_map}
2D to 2D correspondences obtained by dense features matching (section~\ref{subsec:matching}) do not provide enough information to compute relative pose between images at absolute scale. Therefore, we propose to reconstruct the relative scene geometry from the camera to circumvent this limitation. Various recent deep learning generative models are able to properly reconstruct geometry associated to radiometric data, with full supervision training~\cite{Eigen2014}, weakly annotated data~\cite{Godard2017} or even in a self-supervised way~\cite{Mahjourian2018}. 

We train an encoder/decoder jointly to predict the corresponding depth map $\mathrm{M}$ associated to an image: $\mathrm{M = D(E(I))}$. With the generated depth map obtained by our neural network and the intrinsic parameters of the camera $\mathbf{K}$, we can project the 2D point $\left( l, m \right)^T$ to the corresponding 3D coordinate $\mathbf{p}$:
\begin{equation}
	\label{eq:3d_proj}
	\mathbf{p} = \mathrm{M}^{l, m} \cdot \mathbf{K}^{-1}[l, m, 1]^T.
\end{equation}