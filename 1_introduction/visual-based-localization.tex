\section{\Acl*{vbl} with heterogeneous data}

As introduced in the previous section, the main concern of this thesis consist of solving a \ac{vbl} problem. In the following, we introduce the general formulation of \ac{vbl} and the particularity of our problem: the heterogeneity of the visual data.

\subsection{\Acl*{vbl}}
	\Ac{vbl} consists of retrieving the location of a visual query material within a known space representation~\citep{Zamir2016, Piasco2017, Brejcha2017}. For instance, recovering the pose (position + orientation) of a camera that took a given photography according to a set of geo-localized images or a 3D model is a simple illustration of such a localization system~\citep{Kendall2015, Sattler2016a}. \ac{vbl} has been an increasingly dynamic research subject in the last decade. This recent gain of interest is due to the provision of large geo-localized images database, the multiplication of embedded visual acquisition system (\eg camera on smart-phone) and the limitation of usual localization system in urban environment (\eg GPS signal failure in cluttered environment). Aforementioned localization problem is involved in several present-day practical applications, such as GPS-like localization system~\citep{Armagan2017b}, indoor~\citep{Cavallari2018} or outdoor navigation~\citep{Brahmbhatt2017}, 3D reconstruction, models and databases update, cultural heritage~\citep{Bhowmik2017}, consumer photography~\citep{Hays2008, Weyand2016} and augmented reality~\citep{Glocker2013}. \Ac{vbl} is also used in robotics to solve SLAM loop-closure problem~\citep{Garg2018a} or «kidnapped robot» scenario~\citep{Cupec}.
	
	\Ac{vbl} is a very challenging problem. The main obstacle comes from the fact that the visual request we want to localize have been taken at a different time than the database. Visual changes may occur on the observed environment during this period of time, especially if we target outdoor localization~\citep{Lowry2016, Sattler2017a}. For outdoor \ac{vbl}, the appearance of the same scene observed from the query and the reference data can be different due to season changes~\citep{Krajnik2017a}, day-night cycle~\citep{Porav2018}, weather conditions~\citep{Porav2019}, mobile objects~\citep{Toft2018} (like cars or pedestrian) or urban evolution~\citep{Saha2018} (\eg destruction or creation of buildings, change of street furniture). For the indoor case~\citep{Taira2018}, visual changes can be generated by the modulation of the lightning conditions~\citep{Lu2016}, a rearrangement of a room, the people occupancy, etc. Differences in the request and the reference are also observable when the condition of the data acquisition differ. This can be due to the sensor architecture, \eg database and reference images acquired by a different cameras~\citep{Middelberg2014, Majdik2013} or to differences on the pose of the agent that acquire the data, resulting on important view point changes~\citep{Majdik2013, Torii2011, Lin2013, Vo2016, Tian2017}. It is very challenging to address these corner case with image-only \ac{vbl}. However, the use of over information, such as the scene geometry~\citep{Uy2018, Yew2018} or a semantic understanding of the image~\citep{Weinzaepfel2019}, can circumvent the limitation of mono-modal \ac{vbl}.
		
\subsection{Heterogeneous data in \acs{vbl}}
	\Ac{vbl} involves comparing a visual data for which we seek the location, the \textbf{query}, to a geo-located reference database. In conventional pipeline, the query and the reference data are from the same modalities: \textit{e.g} an image and a collection of geo-referenced images~\citep{Arandjelovic2014, Arandjelovic2017} or a segment of a 3D model with semantic information to a the fully semantized 3D model~\citep{Schonberger2017a}. In this work, we are interested in \ac{vbl} with heterogeneous data, \ie the query and the reference data may not contain the same modalities. As mentioned in \acl{sec}~\ref{sec:thesis_env}, in the \ac{plat} project we are interested in localization of a heterogeneous visual query to a set of multi-modal geo-localized RGBDL spheres. That means we could encounter missing modalities or missing data within the queries comparing to the radiometric, depth and semantic modalities present in the reference data. 
	
	It exists plenty of \ac{vbl} methods relying on a single-modality such as radiometric information~\citep{Liu2018, Radenovic2017}, geometric information~\citep{Uy2018, Yew2018} or semantic information~\citep{Ardeshir2014} as well as methods based on multi-modal data: images and geometry~\citep{Schonberger2017a}, images and semantic~\citep{Arandjelovic2014a}, etc. The principal challenge comes when we observe an asymmetric representation of modalities between the request and the reference data. How to compare, or combine, data from different nature? This is a complex question, consequently, there is lack of method that benefit from heterogeneous data for the task of localization. We believe that heterogeneous and asymmetric data can be used wisely for the task of \acl{vbl} to overcome the limitation of single-modality systems. Therefore, within this thesis, we pursue research in this direction.