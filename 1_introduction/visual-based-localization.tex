\section{\Acl*{vbl} with heterogeneous data}

As introduced in the previous section, the main concern of this thesis consist of solving a \ac{vbl} problem. In the following, we introduce the general formulation of \ac{vbl} and the particularity of our problem: the heterogeneity of the visual data.

\subsection{\Acl*{vbl}}
	\Ac{vbl} consists of retrieving the location of a visual query material within a known space representation~\citep{Zamir2016, Piasco2017, Brejcha2017}. For instance, recovering the pose (position + orientation) of a camera that took a given photography according to a set of geo-localized images or a 3D model is a simple illustration of such a localization system~\citep{Kendall2015, Sattler2016a}. \ac{vbl} has been an increasingly dynamic research subject in the last decade. This recent gain of interest is due to the provision of large geo-localized images database, the multiplication of embedded visual acquisition system (\eg camera on smart-phone) and the limitation of usual localization system in urban environment (\eg GPS signal failure in cluttered environment). Aforementioned localization problem is involved in several present-day practical applications, such as GPS-like localization system~\citep{Armagan2017b}, indoor~\citep{Cavallari2018} or outdoor navigation~\citep{Brahmbhatt2017}, 3D reconstruction, models and databases update, cultural heritage~\citep{Bhowmik2017}, consumer photography~\citep{Hays2008, Weyand2016} and augmented reality~\citep{Glocker2013}. \Ac{vbl} is also used in robotics to solve SLAM loop-closure problem~\citep{Garg2018a} or «kidnapped robot» scenario~\citep{Cupec}.
		
\subsection{Heterogeneous data}
	\Ac{vbl} involves comparing a visual data for which we seek the location, the \textbf{query}, to a geo-located reference database. In conventional pipeline, the query and the reference data are from the same modality: \textit{e.g} an image and a collection of geo-referenced images~\citep{Arandjelovic2014, Arandjelovic2016} or a segment of a 3D model with semantic information to a the full semantized 3D model~\citep{Schonberger2017a}. In this work, we are interested in \ac{vbl} with heterogeneous data, \ie the query and the reference data may not contain the same modalities. As mentioned in \acl{sec}~\ref{sec:thesis_env}, in the \ac{plat} project we are interested in localization of a heterogeneous visual query to a set of multi-modal geo-localized RGBDL spheres. That means we could encounter missing modalities or missing data within the queries comparing to the radiometric, depth and semantic modalities present in the reference data.