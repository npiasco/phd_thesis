\section{\Acl*{vbl} with heterogeneous data}

In this section, we introduce the notion of \ac{vbl} as it is the main concern of this thesis.

\subsection{\Acl*{vbl}}
	
	\Ac{vbl} consists of retrieving the location of a visual query material within a known space representation. For instance, recovering the pose of a camera that took a given photography according to a set of geo-localized images or a 3D model is a simple illustration of such a localization system. \ac{vbl} has been an increasingly dynamic research subject in the last decade. This recent gain of interest is due to the provision of large geo-localized images database, the multiplication of embedded visual acquisition system (\textit{e.g}. camera on smart-phone) and the limitation of usual localization system in urban environment (\textit{e.g.} GPS signal failure in cluttered environment). Aforementioned localization problem is involved in several present-day practical applications, such as GPS-like localization system, indoor or outdoor navigation, 3D reconstruction, models and databases update, cultural heritage, consumer photography---``Where did I take these photos?''---and augmented reality. \Ac{vbl} is also used in robotics to solve SLAM loop-closure problem or «kidnapped robot» scenario.
		
\subsection{Heterogeneous data}

	
	