\section{Challenges}
Two main challenges: all \ac{vbl} system. The proposed method have to be accurate, fast and robust. 
Second challenge handling multimodal and heterogeneous data within the localization process. That mean benefit from all the sources for the specific task of localization and find method to compare data of different natures.

\section{Thesis outlines}
This thesis presents an original research work on the development of a new \ac{vbl} method taking benefit from heterogeneous data. An exhaustive review of existing \ac{vbl} methods is presented in \acl{chp}~\ref{chap:2}, with a special attention paid to data heterogeneity within these methods. \Ac{chp}~\ref{chap:3} presents a learned global image descriptor augmented with auxiliary modalities and designed for the task of \ac{vbl}. In \ac{chp}~\ref{chap:4} we introduce a original image pose refinement step which uses missing geometric information to improve localization performances over our first method. Finally, \ac{chp}~\ref{chap:5} concludes the thesis and offers avenues for prospective work.