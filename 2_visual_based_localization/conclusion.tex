\section{Conclusion}

In this chapter, we have been through the principals characteristic of a \ac{vbl} system. We began by presenting common data description shared by localization method, then we introduced the three main classes of localization methods, \ac{cbir} for localization, 6-\ac{dof} pose estimation and coarse to fine localization, and we reviewed the principal \ac{vbl} systems within these classes. In the second part of this chapter, we described the major challenges encountered in real-condition \ac{vbl} scenarios and, in a second step, we have shown how auxiliary information about the scene, like the geometry or the semantic, can circumvent the limitation induced by the use of only radiometric data.

During the discussion, we have highlighted two major trends in modern localization systems. On the application level, the main challenge concerns the long-term localization scenario, where the queries and the reference data can be very different. From a methodological point of view, 2-step localization methods (also named localization cascade or coarse to fine localization) are providing the best trade-off between precision and coverage. That is why we decided to focus our research on the design of a 2-step \ac{vbl} system well suited for long-term localization. In order to do so, our method will take advantages of learned geometric clues from a modality-transfer \ac{cnn}. We have paid a particular attention at the implementation of our method. Thus, the proposed localization system is light and do not rely on a heavy scene representation: it is therefore suitable for embedded or robotic applications. The two following chapters, chapter~\ref{chap:3} and chapter~\ref{chap:4}, are receptively dedicated to our new image descriptor for localization and to our pose refinement step based on learned geometry.



