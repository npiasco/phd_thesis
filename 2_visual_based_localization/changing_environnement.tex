\section{Data with Dissimilar Appearances}
\label{sec:changing_environment}	

	\begin{figure}[t]
	\centering
    \begin{minipage}{0.44\linewidth}
    		\centering
   			\includegraphics[width=\linewidth]{changes/viewpoint.png}

			\includegraphics[width=\linewidth]{changes/daynight2.png}
			
			\subfigure[][Appearance changes]{\label{fig:changes}\includegraphics[width=\linewidth]{changes/shadow.png}}
    \end{minipage}
	\hfill
	\begin{minipage}{0.53\linewidth}
   		\subfigure[][Cross-view]{\label{fig:cross-view}\includegraphics[width=\linewidth]{changes/cross-view.png}}
   		    
   		\subfigure[][Cross-domain]{\label{fig:cross-domain}\includegraphics[width=\linewidth]{changes/cross-domain.png}}
	\end{minipage}
	\caption[Illustration of appearance changes present in \acs*{vbl} system]{\textbf{Illustration of appearance changes present in \ac{vbl} system:} \ref{fig:changes}~Visual dissimilarity between the query (left) and the closest image in the database (right). Cause of the change, from top to bottom:~viewpoint differences~\citep{Majdik2013}, daytime to nighttime image matching~\citep{Porav2018} and shadow interferences from~\citep{Corke2013}. \ref{fig:cross-view}~Cross-view localization system~\citep{Lin2015}:~left represent the ground-level query image and right the bird's eye view of the same scene. \ref{fig:cross-domain}~Cross-domain \ac{vbl} system~\citep{Aubry2014}:~on the left the query painting and on the right the corresponding pose according to a 3D model. \label{fig:data_changes}}
\end{figure}


	As pointed out by \citet{Lowry2016}, permanent changes occurring in our environment is a huge concern in vision domain. In \ac{vbl}, to the difference of SLAM based navigation methods \citep{Garcia-Fidalgo2015,Lowry2016}, the environment representation (i.e. the database) is most of the time acquired at a single date and query can be opposed to the system years after. To take into account local changes of the environment the database needs to be updated. Depending on the size of the covered area, database update can be a costly operation. Thus, an ideal \ac{vbl} system should be able to handle minor visual changes from various sources: daily and season cycle, difference in viewpoint or modifications of the local geometry of the scene. In this section we review selected \ac{vbl} papers that tackle the problem of visual changes in the environment. We dedicate the second part of the section to localization methods that consider extreme appearance changes between the query and the database, namely: cross-view and cross-domain \ac{vbl} systems.
	
	\subsection{Appearance changes}
	\label{subsec:appearance}
		\paragraph{Viewpoint changes}		
			\label{para:viewpoint}
			Common visual acquisition systems capture a part of the environment lying inside the frustum of the sensor. Indeed, camera are oriented-device and due to the complex geometry of our surrounding environment, viewpoint changes in visual data impact drastically the appearance of the same scene. To handle those changes, local descriptors described in \S\ref{subsec:local_feature} have been widely used. By describing partial areas of the whole scene, local features are naturally robust to a certain amount of changes introduced by difference in viewpoint, small occlusion or scene modification. \citet{Wan2014} treat extreme viewpoints changing (when the camera are facing each other) in repetitive lunar environment. To achieve \ac{vbl} in such conditions they match the ground part of the image (which is subject to large affine distortion) with a fully affine invariant feature~\citep{Morel2009}.
			
			Image rectification~\citep{Forstner2016} is also employed in \ac{vbl} to minimize appearance changes introduced by different viewpoints. With strong assumption on the environment where the localization is performed (\textit{e.g.} such as Manhattan world assumption~\citep{Murillo2013,Cham2010}), images rectification ensure that facing direction of all visual data will be barely the same. With the hypothesis of an urban scene, vanishing points can be extracted~\citep{Lezama2014,Hartley2003,Forstner2016} and images rectified to display front facing buildings~\citep{Robertson2004,Chen2011,Morago2016,Arth2015,Cham2010}.
			
			Other approaches consists of filling the database with additional data to cover all the possible viewpoints for a given environment. \citet{Milford2015} generate translated view on a database road-circuit for preventing miss-matches if the car, carrying the acquisition system, is moving on a different traffic way than the one used to collect the database. Notice the use of a \ac{cnn} depth estimator from mono image in order to produce consistence synthetic shifted-views. Work from~\citep{Irschara2009,Aubry2014,Torii2015} increase the number of documents in the database by automatic data generation to ensure that whatever the viewpoint of an incoming query, a document displaying a similar view can be retrieved. \citet{Majdik2013} perform air-ground matching of picture taken by a Micro Air Vehicle (MAV) against street view images (top of figure~\ref{fig:changes}). The main challenge outlined in this paper is the large difference in angle viewpoint. Authors generate artificial view from both the database and the query image to handle the affine transformation introduced by altitude differences (inspired by the work of~\citep{Morel2009}).			
			
		\paragraph{Illumination invariance \& long-term localization}
        	\label{para:illum}
			\citet[Section VII]{Lowry2016} explore exhaustively Visual SLAM methods that perform strong illumination invariance place recognition (\textit{e.g.} SeqSLAM~\citep{Milford2012,Pepperell2014,Pepperell2016} or FAB-MAP~\citep{Cummins2008,Cummins2010,Paul2010}). Illumination perturbation are caused by three main phenomena: weather conditions and illumination changes across season, daily cycle and finally shadow casting (see figure~\ref{fig:changes} for illustration). In~\citep{Lowry2016a}, authors present an invariant-free image representation in order to overcome aforementioned perturbation in visual domain. In a more robotic-oriented-scenario, \citet{Muhlfellner2015} investigate map invariance representation when multiple instances of the same environment are available. 
		
			\begin{description}
				\item[Seasons \& Weather.]
					Illumination changes are usually handled at the beginning of the \ac{vbl} pipeline, during the data description step. Local features robust to illumination, like SIFT or SURF, consider gradient quantity in order to be invariant to pixel intensity variation caused by different illumination conditions. However, \citet{Valgren2010} have shown that these representations are not well suited for similarity association across season cycles. GRIEF local descriptor~\citep{Krajnik2017a} (derivatives of BRIEF~\citep{Calonder2010}) or ORB feature~\citep{Griffith2017} show better results for this task. The use of heterogeneous databases (i.e. composed of data acquired by different supports, see \S\ref{para:data_consistency}) constrain the system to be robust to disparate illumination conditions~\citep{Arandjelovic2017}. Works described in~\citep{Krajnik2014,Krajnik2017a} model seasonal-like cycle in a probabilistic framework in order to downgrade features that are not likely to appear during a given period of time. \citet{Rosen2016} also propose a model to take in account the features persistence, decreasing the probability of encountering a feature that have been met for the first time a long time ago.  On the other hand, learned descriptors show good performances if trained for the specific inter-season matching task~\citep{Carlevaris-Bianco2014}.

				\item[Nocturnal illumination.]
					In some application, especially for vehicle localization, \ac{vbl} has to be performed during a complete day, including overnight~\citep{McManus2014,Milford2015} (see middle example of figure~\ref{fig:changes}). Dense descriptors'~extraction used in \citep{Torii2015} exhibit promising result for daytime to overnight images matching. At first glance, artificial lights ubiquitous in urban scene can be considered as sources of disruption. However, \citet{Nelson2015} focus on this particular clues to perform localization across only night road images.
				
                \item[Shadows.]
					Some researches focus on the specific perturbation introduced by shadow casting over images. \citet{Wan2016} outline that satellite and overhead images can change drastically in appearance depending on the relative position of the sun during the day. Authors show that Fourier transforms can be used to create shadow-invariant image representation. \citet{Corke2013} implement the shadow suppression method presented in~\citep{Finlayson2006} to localize street images with important depth artefacts projected by trees or buildings. This method still remains very sensor-dependent.
			\end{description}

		\paragraph{Dynamic scene}
			As mentioned previously, methods based on local descriptors are prompt to handle local changes in images due to dynamic modifications of the environment (\textit{e.g.} vegetation growing, buildings construction or annihilation, presence of pedestrians or vehicles, partial occlusions, etc.). Several investigations have been led for designing robust descriptors to local geometric changes. \citet{Kim2015} train SVM classifiers to discriminate strong and weak local features for the \ac{vbl} task. The method shows promising results where features are more often selected when they are attached to persistent objects, such as facades, and dismissed when they represent ephemeral or changing elements, such as people or trees. Based on similar observation, \citet{Mousavian2015} introduce down-weighting of irrelevant features according to their semantic class. Learning approaches have also been investigated in other works. \citet{Arandjelovic2017} train a \ac{cnn} for global description upon images from the Google Street View Time Machine to get diverse representation of the same scene captured over a period of ten years. From this kind of representation of the environment, persistent clues can be efficiently extracted~\citep{Neubert2015}. Similarly, \citet{Kumar2016mastersThesis} proposes a CNN approach for place recognition across seasons.

	\subsection{Cross-appearance localization}
	\label{subsec:cross_domain}
		Subsequent part focus on methods that reach an extreme with change-invariance consideration by creating cross-appearance algorithms for \ac{vbl}. We distinguish between two main categories of applications: cross-view \ac{vbl}, where authors localize a ground-view image against database of aerial images, and cross-domain \ac{vbl}, where the purpose is to localize an image of a certain nature within a database of different nature.
		
		\paragraph{Cross-view}
			\label{para:cross_view}
			Cross-view localization, also denoted as ultra-wide baseline matching~\citep{Bansal2012}, consider the problem of ground level localization from aerial-level set of photo shoots (see figure~\ref{fig:cross-view} for an illustration of data association targeted by cross-view systems). Cross-view \ac{vbl} is motivated by the fact that satellite photographies are rich sources of information, available almost all over the globe. However, finding similarity between data acquired at a ground level and data captured with flying devices is a hard task due to the extreme change in viewpoint. A series of works consider cross-view localization~\citep{Lin2013,Workman2015,Castaldo2015,Vo2016,Tian2017}. In~\citep{Workman2015,Vo2016}, authors investigate the use of a CNN to automatically associate ground level images taken from street view service with fine-grained overhead images. \citet{Vo2016} compare several CNN architectures and conclude that triplet trained network provides the most suitable descriptors for cross-view matching. Rotation invariance between ground and overhead images is also studied through auxiliary loss and special training. 
			
			In~\citep{Bansal2011,Bansal2012,Lin2015}, authors use bird's eye imagery to localize ground level snapshots. \citet{Bansal2011} method relies on ground level images rectification, like methods focused on viewpoint changes (refer to \S\ref{para:viewpoint}).
			
		\paragraph{Cross-domain}
			\label{para:cross_domain}
			Another field of research where the data association is very challenging is the cross-domain localization (an example of cross-domain \ac{vbl} is presented in figure~\ref{fig:cross-domain}). \citet{Russell2011} work, followed by \citet{Aubry2014} contribution, focus on the task of retrieving the pose of an old hand-drafted document (a sketch or a painting) according to a known realistic representation. In~\citep{Aubry2014}, hard training of HOG-based descriptors are used to capture the global shape of the architectural scene displayed in the documents, in the same manner as~\citep{Shrivastava2011}. Results are impressive, but the used descriptor is not robust to viewpoint changes. Cross-domain techniques are also used to recover the pose of ancient photographies and to confront them with current data \citep{Bae2010,Bhowmik2017}.