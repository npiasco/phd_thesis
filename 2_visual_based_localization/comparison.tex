\section{Discussion}
\label{sec:comparison}
	This section aims to highlight common usage and emerging trends in \ac{vbl}. As \ac{vbl} panorama is wide and varied, we first propose a review of recent datasets and evaluation metrics used for comparing different approaches.
    
	\subsection{Datasets}
		\paragraph{Coverage.}			
		By definition, the previously described applications do not cover the same area. For instance, a system design for car pose estimation should be able to localize a vehicle in a larger area than a pedestrian \ac{vbl} system should do. Thus, there exists systems designed for world-wide localization estimation~\citep{Hays2008,Weyand2016,Vo2017} as well as more spatially focused system~\citep{Song2016}. Database coverage can be extended by using wide angle or omnidirectional cameras. Works presented in~\citep{Arandjelovic2017,Iscen2017,Kumar2016mastersThesis,Ramalingam2010,Torii2011,Zamir2010,Zamir2014} use databases composed of spherical panoramas. On the other hand, aerial images are likely to cover large area but restrict \ac{vbl} applications~\citep{Wan2016}.
		
		\paragraph{Database Consistency.}
		\label{para:data_consistency}
		Regardless of the type of images used, we distinguish between two kinds of databases: homogeneous and heterogeneous. Homogeneous databases are composed of images gathered through the same optical acquisition system in a restricted time interval. Google street view\footnote{https://www.google.fr/intl/usa/streetview/} or \textit{IGN St\'er\'eopolis} platform~\citep{Paparoditis2012} give access to this kind of database. Homogeneous databases are suitable for applications that perform systematic processing on the data~\citep{Majdik2013,Torii2015}. Conversely, heterogeneous databases are made of images collected by various people with different cameras at inconsistent periods. These databases are often constructed through on-line collaborative platforms, like flickr\footnote{https://www.flickr.com/}, by downloading images associated to specific tags. Heterogeneous databases have the advantage of introducing visual variations in the \ac{vbl} system and therefore improves robustness to appearance changes~\citep{Radenovic2016,Gordo2016} (see Section~\ref{sec:changing_environment}). Additionally, heterogeneous databases are easier to augment or update compared to homogeneous ones.

		\paragraph{Current datasets} Because of the important difference between direct and indirect methods, there exists two kinds of datasets used in \ac{vbl}: unsorted list of images and spatially consistent datasets (that can be composed of point cloud or fine geo-referenced images). Table~\ref{tab:dataset} presents numerous datasets used in \ac{vbl}. Notice the growing number of publicly available datasets starring complete 3D scans of large cities~[dataset]\citep{Menze2015,Maddern2016,Wang2016}. As mentioned earlier, long-term localization in changing environment is an hot topic in robotic research. We observe therefore appearance of several datasets featuring multiple acquisitions of the same place over long periode of time~\citep{Maddern2016,Carlevaris-Bianco2016,Krajnik2010,Krajnik2014}.

		\begin{landscape}
\begin{table}[!ht]
	\centering
	\caption[Currently used datasets in \ac{vbl}]{\label{tab:dataset} \textbf{Currently used datasets in \ac{vbl}.} Depending on the method to be evaluated (\textit{e.g.} direct or indirect), different datasets are used. Data information (pose, type and homogeneity) concern the documents composing the database and not the one used at query time. Data homogeneity refer to the definition given in paragraph~\ref{para:data_consistency}. RGB-D refer to data recorded with depth-cameras and RGB-S to information collected with standard cameras coupled with laser-scan depth estimation. $^{\dagger}$~6 DoF available in~\citep{Sattler2017}.}
	\renewcommand{\arraystretch}{1.1}
	\footnotesize{
		\begin{tabular}{l  l  c  c  c}
      		\hline
			\bf Name		& \bf Application domain	& \bf Data Pose Info.		& 	\bf Data Type	& \bf Data Homog.	  \\
      		\hline
      		\hline
			\href{http://lear.inrialpes.fr/people/jegou/data.php\#holidays}{INRIA Holidays} [dataset]\citep{Jegou2008} & Scene retrieval & No & RGB & No \\
			\href{http://www.robots.ox.ac.uk/~vgg/data/oxbuildings/}{Oxford Buildings} [dataset]\citep{Philbin2007} & Landmark retrieval & No & RGB & No \\
			\href{http://www.robots.ox.ac.uk/~vgg/data/parisbuildings/}{Paris} [dataset]\citep{Philbin2008} & Landmark retrieval  & No & RGB & No \\
			\href{http://image.ntua.gr/iva/datasets/wc/}{World Cities Dataset} [dataset]\citep{Tolias2011} & Image retrieval & GPS & RGB & No \\
            \href{http://www.ok.ctrl.titech.ac.jp/~torii/project/repttile/}{Pittsburgh 250k} [dataset]\citep{Torii2013} & Image retrieval & GPS & RGB & Yes \\
            \href{https://purl.stanford.edu/vn158kj2087}{San Francisco Landmark} [dataset]\citep{Chen2011} & Landmark retrieval & GPS$^{\dagger}$ & RGB & Yes \\[3pt]

			\href{http://crcv.ucf.edu/projects/GMCP_Geolocalization/}{Pittsburgh Street View} [dataset]\citep{Zamir2014} & Image retrieval & GPS + Compass & RGB & Yes \\
			\href{http://www.ok.ctrl.titech.ac.jp/~torii/project/247/}{Tokyo 24-7 dataset} [dataset]\citep{Torii2015} & Image retrieval & GPS + Compass & RGB & No \\[3pt]

			\href{https://nrkbeta.no/2013/01/15/nordlandsbanen-minute-by-minute-season-by-season/}{Nordland train dataset} & Inter-season matching & GPS & RGB & No \\
			\href{http://labe.felk.cvut.cz/~tkrajnik/dataset/}{Stromovka dataset} [dataset]\citep{Krajnik2010} & Inter-season matching & Inter-season pair & RGB & No \\[3pt]		

			\href{https://cvg.ethz.ch/research/mountain-localization/}{CH1 dataset} [dataset]\citep{Saurer2016} & Localization in mountain & GPS & RGB & No \\
			\href{https://cvg.ethz.ch/research/mountain-localization/}{CH2 dataset} [dataset]\citep{Saurer2016} & Localization in mountain & GPS & RGB & Yes \\
			\href{http://cphoto.fit.vutbr.cz/geoPose3K/}{GeoPose3K} [dataset]\citep{Brejcha2017a} & Localization in mountain & 6 DoF Pose & RGB & No \\[3pt]
			\href{http://mi.eng.cam.ac.uk/projects/relocalisation/\#dataset}{Cambridge Dataset} [dataset]\citep{Kendall2015} & Camera localization & 6 DoF Pose & SfM & Yes \\
			% Rajouter Vienna dataset (homogen si je m'en souviens bien, je trouve pas le lien vers le DS
			\href{http://www.cs.cornell.edu/projects/p2f/}{Rome16K} [dataset]\citep{Li2010} & Camera localization & 6 DoF Pose & SfM & No \\
			\href{http://www.cs.cornell.edu/projects/p2f/}{Dubrovnik6K} [dataset]\citep{Li2010} & Camera localization & 6 DoF Pose & SfM & No \\
			\href{https://www.graphics.rwth-aachen.de/software/image-localization}{Aachen} [dataset]\citep{Sattler2012a} & Camera localization & 6 DoF Pose & SfM & No \\
			\href{http://phototour.cs.washington.edu/datasets/}{Notre Dame dataset} [dataset]\citep{Snavely2006} & Camera localization & 6 DoF Pose & SfM & No \\[3pt]
			\href{https://www.microsoft.com/en-us/research/project/rgb-d-dataset-7-scenes/}{7 scenes} [dataset]\citep{Shotton2013} & Multi-purpose (indoor) & 6 DoF Pose & RGB-D & Yes \\
			\href{https://strands.pdc.kth.se/public/Witham_Wharf_RGB-D_dataset/}{Witham Wharf dataset} [dataset]\citep{Krajnik2014} & Multi-purpose (indoor) & 6 DoF Pose & RGB-D & No \\		
			\href{http://robots.engin.umich.edu/nclt/}{North Campus dataset} [dataset]\citep{Carlevaris-Bianco2016} & Multi-purpose & 6 DoF Pose & RGB-S & No \\			\href{http://mrg.robots.ox.ac.uk/the-oxford-robotcar-dataset/}{Oxford Robotcar} [dataset]\citep{Maddern2016} & Multi-purpose & 6 DoF Pose & RGB-S & No \\
			\href{?}{TorontoCity dataset} [dataset]\citep{Wang2016} & Multi-purpose & 6 DoF Pose & RGB-S & Yes \\
			\href{http://www.cvlibs.net/datasets/kitti/}{KITTI dataset} [dataset]\citep{Menze2015} & Multi-purpose & 6 DoF Pose & RGB-S & Yes \\
			\hline	
		\end{tabular}
	}
    \footnotetext[4]{6 DoF available in~\citep{Sattler2017}}    
\end{table}
\end{landscape}

		\paragraph{Evaluation Metrics}
		\label{subsec:evaluation_metric}
			Authors use various types of performances criteria in order to compare indirect methods. The recall @$k$, or recall @$k$\%, is the most discriminative metric for \ac{vbl} evaluation. It represents the percentage of queries that present a good match within the $k$ or $k$\% top ranked images. Usually $k$ is set to 10 or 1\%. Classical object-retrieval metric can be used, like RoC curves (precision against recall), mAP (the mean of average precision value) or the simple recall rate. If images in the database are augmented with GPS tag, authors often decide that a query is correctly localized if one among $k$ retrieved candidates lies inside a tolerance radius (usually 10m, depending on the dataset). Result visualization is obtained by plotting a variable number $k$ of candidates against the fraction of correctly localized images.
			
			Concerning direct methods, authors often simply compute the mean of absolute position and orientation error relative to the available ground truth. Another criterion can be extracted from the inlier count obtained against a robust geometric verification. A query is considered as successfully matched if enough inliers are found after the application of an iterative RANSAC-like algorithm. However, such a metric does not ensure that the data is well localized according to the model~\citep{Sattler2015}.
			
	\subsection{Runtime consideration}
		\label{subsec:runtime}
		Real-time performances and embedded architectures are constraints mainly present in the robotic community. In \ac{vbl}, such criteria are not always taken into account. This can be explained by the fact that recovering the localization of an input query is a one-shoot action; i.e. it has to be performed only once compared to tracking systems~\citep{Marchand2016} or SLAM algorithms~\citep{Garcia-Fidalgo2015}. Furthermore, more and more embedded systems rely on deported architecture or cloud computer, resolving the problem of low computational power of portable devices~\citep{Middelberg2014}. Yet, some authors manage to reduce the computational cost of their system~\citep{Shotton2013,Glocker2015,Lynen2015}: for instance \citet{Feng2016a} introduce a light version of F2P method and works from~\citep{Weyand2016,Kendall2015,Contreras2017} embed their localization system in a compact CNN architecture loadable on a smart-phone.
			
	\subsection{Trends in VBL}
		\label{subsec:qualitative_comparison}
    	
		A quantitative comparison between all \ac{vbl} systems is impossible due to the diversity in both methods and applications. Nevertheless, we refer reader to recent papers that quantitatively compare specific types of state-of-the-art methods. Concerning indirect methods, following recent contributions~\citep{Radenovic2016,Gordo2016} show comprehensive comparisons. Direct F2P methods based on \ac{sfm} are carefully compared on three papers~\citep{Feng2016a,Sattler2016a,Svarm2016} and \citet{Kendall2017} propose a detailed comparison between \ac{sfm}-direct methods (\S\ref{subsec:sfm_methods}) and \ac{cnn}-direct methods (\S\ref{para:cnn_regressor}). In \citep{Brejcha2017}, authors propose an overview of both indirect and direct \ac{vbl} methods by reporting results of various works in a common table. Finally, \citet{Sattler2017} present the first comparison between indirect and direct methods based on images collection for the task of query accurate localization. In the following, we propose our qualitative analysis of \ac{vbl} panorama. 
        
        \paragraph{From indirect to direct methods}
        	As discussed earlier, there is a trade-off between the area covered and the precision reached by the \ac{vbl} system; the survey of \citet{Brejcha2017} provides a complete overview of this problem. Indirect methods prioritize the space coverage, city scale~\citep{Gordo2016} to word-wide~\citep{Weyand2016}, whereas direct methods focus on precision and exact 6 \ac{dof} estimation~\citep{Feng2016a}. This survey describes \ac{vbl} methods in a chronological order, that is why we present indirect methods before direct ones. However, during the last decade, research focus seems to have turned to direct methods. This can be explained by the drastic increase of applications using precise~\ac{vbl} for both professional (\textit{e.g.} robotics~\citep{Majdik2013}) and individual (\textit{e.g.} augmented reality~\citep{Arth2015}) purposes.
        
        \paragraph{The growing importance of geometric data}
        	Limitations of methods using only images have been discussed in Section~\ref{sec:changing_environment}, and the growing accessibility of geometric data promotes the development of systems based on depth information~\citep{Paparoditis2012}. Furthermore, geometric data facilitates the final pose estimation, which also explains the rapid development of direct methods~\citep{Kroeger2014,Pani2015Lmi,Pani2015Robust}. When available, depth information can directly improve the result of \ac{vbl} methods~\citep{Ni2009,Gee2012,Shotton2013,Torii2015,Cavallari}. Point clouds remain the favourite type of geometric data in~\ac{vbl}~\citep{Sattler2016a}, nevertheless less complete but more compact models have shown promising result for some applications~\citep{Ramalingam2010,Bansal2014,Christie2016,Torii2015}.

        \paragraph{Emergence of semantic localization}
        	Less used in \ac{vbl}, semantic data offer promising results~\citep{Ardeshir2014,Castaldo2015,Christie2016}. In addition to being generic regarding the original ``raw'' scene representation, semantic abstraction permits a discriminative and robust description of the scene. Moreover, the use of semantic graph representations can handle huge amount of data. Loss in area coverage granted by the use of direct methods and complex data can be balanced by semantic information. Indeed, in \citep{Ardeshir2014,Lu2015} authors initially narrow the research scope with extracted semantic clues.
        
        \paragraph{Data combination and cascade schemes}
        	Getting both wide area coverage and high precision of the query pose is the current challenge of \ac{vbl}. Cascade scheme, that can be seen as a combination of indirect and direct methods, are certainly a good alternative to achieve this objective~\citep{Sattler2017}. Indeed, firstly reducing the amount of data and in a second step recovering the exact pose of the query is a well studied topic~\citep{Rubio2015,Azzi2016,Song2016,Meng2016,Sattler2017}. This architecture facilitates the use of heterogeneous data~\citep{Lu2015} in a common framework. Combination of various types of data benefits to the task of \ac{vbl} by exploiting all the available sources of interest present at a given location~\citep{Li2016}.
            
	\subsection{Benefit of heterogeneous data}
    	All along this survey we emphasize the growing importance of multiple types of data (optical, geometric and semantic information) for the task of \ac{vbl}. As discussed in section~\ref{sec:application}, using more sophisticate data aims to overcome shortcoming (presented in section~\ref{sec:changing_environment}) of only-optical based systems. Geometric and depth information permit a total abstraction to the pixel intensity, naturally providing to the system a robustness to illumination variability that can occur in the scene. Local changes in scene appearance and geometry can also be handle with the use of a semantic representation. Optical data, though, contain extremely specific information and are much more easier to collect compared to semantic and geometric data. That is why these three aforementioned data should be considered as complementary information. Based on these observations, there is a real benefit of using heterogeneous data to achieve better results in \ac{vbl}.
		
	\subsection{VBL and machine learning} 
    	\ac{vbl} benefits from the recent progress in machine learning. Indirect methods are now dominated by \ac{cnn} approaches~\citep{Radenovic2016,Gordo2016} (see \S\ref{para:global_cnn}). Image description obtained with convolutional networks gathers all the characteristics needed for the task of scene retrieval. New network architecture has been proposed to resolve the direct \ac{vbl} problem (presented in \S\ref{para:cnn_regressor}). Despite the simplicity and the robustness of these methods, state-of-the-art direct localization results are still obtained through point to feature approaches~\citep{Walch2016a}. However, active researches are pursuing in this promising direction~\citep{Liu2016,Jia2016,Kendall2017}. The last \ac{vbl} sub-domain improved by \ac{cnn} is the semantic scene segmentation and categorization used in some localization methods~\citep{Salas-Moreno2013,Ardeshir2014,Arth2015,Christie2016}. Conventional methods, like Deformable Parts Model (DPM)~\citep{Ardeshir2014}, SVM~\citep{Arth2015} or Automatic Labelling Environment (ALE)~\citep{Christie2016}, should be quickly replaced by CNN~\citep{Sunderhauf2015a,Zhao2016}.