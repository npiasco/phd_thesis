\section{Discussion}
\label{sec:comparison}
	As \ac{vbl} panorama is wide and varied, we first propose a review of recent datasets and evaluation metrics used for comparing different approaches. Afterwards, we highlight common usage and promising research avenues in \ac{vbl}.
	
	\subsection{Datasets}
		Commonly used datasets in \ac{vbl} are presented in table~\ref{tab:dataset}. Because of the important difference between 6-\ac{dof} pose estimation and \ac{cbir} for localization, there exists two kinds of datasets used in \ac{vbl}: list of images (with basic position information or landmark/place tags) and strongly structured datasets (that can be composed of point cloud or fine geo-referenced images). Notice the growing number of publicly available datasets starring complete 3D scans of large cities~\citep{Menze2015,Maddern2016,Wang2016}. As mentioned earlier, long-term localization in changing environment is an hot topic in robotic research. We observe therefore appearance of several datasets featuring multiple acquisitions of the same place over long period of time~\citep{Maddern2016,Carlevaris-Bianco2016,Krajnik2010,Krajnik2014}.
		
		For landmarks recognition in city (used method come mainly from \ac{cbir} for localization approaches), the most used dataset is the revisited version of Oxford and Paris landmarks~\citep{Radenovic2018}. Concerning precise 6-\ac{dof} pose estimation under changing condition, researchers prefer the recent benchmark from \citet{Sattler2018} that compiles multiples datasets focused on long-term localization scenario.
		
		\paragraph{Coverage and consistency.}
		\label{para:coverage_consistency}
		All the application relying on \ac{vbl} do not cover the same area. For instance, a system design for car pose estimation should be able to localize a vehicle in a larger area than a pedestrian \ac{vbl} system should do. Thus, there exists dataset with a coverage spreading from small indoor scenes to world-wide area.
	
		Visual content within a dataset can be uniform, reference and queries are from the same sensor and captured at the same time, or inconsistent, queries have been taken under different condition or with different sensor than the references data. In order to reflect real-life conditions, outdoor datasets are often inconsistent, with visual dissimilarity between queries and references induced by acquisition sensors, dynamic changes in the scene (cars \& pedestrians),  weather conditions etc. Indoor datasets use more uniform data~\citep{Shotton2013}, thus targeting application like camera re-localization for robot navigation or augmented reality.

		\paragraph{Evaluation metrics.}
		\label{subsec:evaluation_metric}
			Authors use various types of performances criteria in order to compare \ac{cbir} for localization methods. The recall @$k$, or recall @$k$\%, is the most used metric. It represents the percentage of queries that present a good match within the $k$ or $k$\% top ranked images. A query is considered correctly localized if it lies inside a tolerance radius from its ground truth position (from 10 to 25m, depending on the dataset).  Usually $k$ is set to 10 or 1\%. If we consider only the top 1 retrieved candidate for evaluation, distance from the ground truth is used as the main precision criteria. For places or landmarks recognition tasks~\citep{Radenovic2018}, \ac{map} evaluate performances of the method.
			
			Concerning 6-\ac{dof} pose estimation methods, authors simply compute the median (rarely the mean) of absolute position and orientation error relative to the ground truth. Another criterion can be extracted from the inlier count obtained against a robust geometric verification (for image based localization). A query is considered as successfully matched if enough inliers are found after \ac{ransac}. However, such a metric does not ensure that the data is well localized according to the model~\citep{Sattler2015}. Percentage of well localized images, \ie under a given error threshold (\eg 5cm \& 5$^{\circ}$ for indoor scene), is also a current evaluation metric.  Recently, \citet{Sattler2018} introduce a more detailed metric by considering multiple level of precision (from fine to coarse localization) with different thresholds.
			
	 \begin{landscape}
\begin{table}[!ht]
	\centering
	\caption[Currently used datasets in \ac{vbl}]{\label{tab:dataset} \textbf{Currently used datasets in \ac{vbl}.} Depending on the method to be evaluated (\textit{e.g.} direct or indirect), different datasets are used. Data information (pose, type and homogeneity) concern the documents composing the database and not the one used at query time. Data homogeneity refer to the definition given in paragraph~\ref{para:data_consistency}. RGB-D refer to data recorded with depth-cameras and RGB-S to information collected with standard cameras coupled with laser-scan depth estimation. $^{\dagger}$~6 DoF available in~\citep{Sattler2017}.}
	\renewcommand{\arraystretch}{1.1}
	\footnotesize{
		\begin{tabular}{l  l  c  c  c}
      		\hline
			\bf Name		& \bf Application domain	& \bf Data Pose Info.		& 	\bf Data Type	& \bf Data Homog.	  \\
      		\hline
      		\hline
			\href{http://lear.inrialpes.fr/people/jegou/data.php\#holidays}{INRIA Holidays} [dataset]\citep{Jegou2008} & Scene retrieval & No & RGB & No \\
			\href{http://www.robots.ox.ac.uk/~vgg/data/oxbuildings/}{Oxford Buildings} [dataset]\citep{Philbin2007} & Landmark retrieval & No & RGB & No \\
			\href{http://www.robots.ox.ac.uk/~vgg/data/parisbuildings/}{Paris} [dataset]\citep{Philbin2008} & Landmark retrieval  & No & RGB & No \\
			\href{http://image.ntua.gr/iva/datasets/wc/}{World Cities Dataset} [dataset]\citep{Tolias2011} & Image retrieval & GPS & RGB & No \\
            \href{http://www.ok.ctrl.titech.ac.jp/~torii/project/repttile/}{Pittsburgh 250k} [dataset]\citep{Torii2013} & Image retrieval & GPS & RGB & Yes \\
            \href{https://purl.stanford.edu/vn158kj2087}{San Francisco Landmark} [dataset]\citep{Chen2011} & Landmark retrieval & GPS$^{\dagger}$ & RGB & Yes \\[3pt]

			\href{http://crcv.ucf.edu/projects/GMCP_Geolocalization/}{Pittsburgh Street View} [dataset]\citep{Zamir2014} & Image retrieval & GPS + Compass & RGB & Yes \\
			\href{http://www.ok.ctrl.titech.ac.jp/~torii/project/247/}{Tokyo 24-7 dataset} [dataset]\citep{Torii2015} & Image retrieval & GPS + Compass & RGB & No \\[3pt]

			\href{https://nrkbeta.no/2013/01/15/nordlandsbanen-minute-by-minute-season-by-season/}{Nordland train dataset} & Inter-season matching & GPS & RGB & No \\
			\href{http://labe.felk.cvut.cz/~tkrajnik/dataset/}{Stromovka dataset} [dataset]\citep{Krajnik2010} & Inter-season matching & Inter-season pair & RGB & No \\[3pt]		

			\href{https://cvg.ethz.ch/research/mountain-localization/}{CH1 dataset} [dataset]\citep{Saurer2016} & Localization in mountain & GPS & RGB & No \\
			\href{https://cvg.ethz.ch/research/mountain-localization/}{CH2 dataset} [dataset]\citep{Saurer2016} & Localization in mountain & GPS & RGB & Yes \\
			\href{http://cphoto.fit.vutbr.cz/geoPose3K/}{GeoPose3K} [dataset]\citep{Brejcha2017a} & Localization in mountain & 6 DoF Pose & RGB & No \\[3pt]
			\href{http://mi.eng.cam.ac.uk/projects/relocalisation/\#dataset}{Cambridge Dataset} [dataset]\citep{Kendall2015} & Camera localization & 6 DoF Pose & SfM & Yes \\
			% Rajouter Vienna dataset (homogen si je m'en souviens bien, je trouve pas le lien vers le DS
			\href{http://www.cs.cornell.edu/projects/p2f/}{Rome16K} [dataset]\citep{Li2010} & Camera localization & 6 DoF Pose & SfM & No \\
			\href{http://www.cs.cornell.edu/projects/p2f/}{Dubrovnik6K} [dataset]\citep{Li2010} & Camera localization & 6 DoF Pose & SfM & No \\
			\href{https://www.graphics.rwth-aachen.de/software/image-localization}{Aachen} [dataset]\citep{Sattler2012a} & Camera localization & 6 DoF Pose & SfM & No \\
			\href{http://phototour.cs.washington.edu/datasets/}{Notre Dame dataset} [dataset]\citep{Snavely2006} & Camera localization & 6 DoF Pose & SfM & No \\[3pt]
			\href{https://www.microsoft.com/en-us/research/project/rgb-d-dataset-7-scenes/}{7 scenes} [dataset]\citep{Shotton2013} & Multi-purpose (indoor) & 6 DoF Pose & RGB-D & Yes \\
			\href{https://strands.pdc.kth.se/public/Witham_Wharf_RGB-D_dataset/}{Witham Wharf dataset} [dataset]\citep{Krajnik2014} & Multi-purpose (indoor) & 6 DoF Pose & RGB-D & No \\		
			\href{http://robots.engin.umich.edu/nclt/}{North Campus dataset} [dataset]\citep{Carlevaris-Bianco2016} & Multi-purpose & 6 DoF Pose & RGB-S & No \\			\href{http://mrg.robots.ox.ac.uk/the-oxford-robotcar-dataset/}{Oxford Robotcar} [dataset]\citep{Maddern2016} & Multi-purpose & 6 DoF Pose & RGB-S & No \\
			\href{?}{TorontoCity dataset} [dataset]\citep{Wang2016} & Multi-purpose & 6 DoF Pose & RGB-S & Yes \\
			\href{http://www.cvlibs.net/datasets/kitti/}{KITTI dataset} [dataset]\citep{Menze2015} & Multi-purpose & 6 DoF Pose & RGB-S & Yes \\
			\hline	
		\end{tabular}
	}
    \footnotetext[4]{6 DoF available in~\citep{Sattler2017}}    
\end{table}
\end{landscape}
			
	\subsection{Trends in VBL}
		\label{subsec:qualitative_comparison}
    	
		A quantitative comparison between all \ac{vbl} systems is impossible due to the diversity in both methods and applications. Nevertheless, we refer reader to recent papers that quantitatively compare specific types of state-of-the-art methods. Concerning \ac{cbir} methods, following recent contributions~\citep{Radenovic2016,Gordo2016} show comprehensive comparisons. In~\citep{Radenovic2018}, authors benchmark state-of-the-art methods on the Revisited Oxford and Paris dataset. F2P methods based on \ac{sfm} are carefully compared on three papers~\citep{Feng2016a,Sattler2016a,Svarm2016}. A comprehensive comparison between 6-\ac{dof} pose estimation approaches, both learned and structured based, is presented in~\citep{Sattler2019}. We refer readers to the \href{https://www.visuallocalization.net/benchmark/}{online} leader-board of the visual localization benchmark~\citep{Sattler2018} for up-to-date bests methods for pose estimation under challenging conditions. In the following, we propose our qualitative analysis of \ac{vbl} panorama. 
        
        \paragraph{Method development.}
        	As discussed earlier, there is a trade-off between the area covered and the precision reached by the \ac{vbl} system; the survey of \citet{Brejcha2017} provides a complete overview of this problem. \ac{cbir} for localization methods prioritize the space coverage, city scale~\citep{Gordo2016} to word-wide~\citep{Vo2017}, whereas full pose computation methods focus on precision and exact 6-\ac{dof} estimation~\citep{Feng2016a} on reduced area. 

	        Getting both wide area coverage and high precision of the query pose is the current challenge of \ac{vbl}. As shown in~\citep{Sattler2017}, coarse to find localization systems (see section~\ref{subsec:vbl_prior}) are certainly a good alternative to achieve this objective. By firstly reducing the amount of data and in a second step recovering the exact pose of the query is a clever manner to target both pose precision and scalability. This cascaded localization pipeline is a well studied research area~\citep{Rubio2015,Azzi2016,Song2016,Meng2016,Sattler2017}, and more and more recent works address the location problem in this way~\citep{Sarlin2018a,Sarlin2018,Germain2019,Taira2018,Taira2019}.
     
     	\paragraph{Benefit of heterogeneous data.}
	     	All along this survey we emphasize the growing importance of multiple types of data (radiometric, geometric and semantic information) for the task of \ac{vbl}. As discussed in section~\ref{sec:application}, using more sophisticate data aims to overcome shortcoming of radiometric based systems.
	     	Geometric data improve the final pose estimation~\citep{Pani2015Lmi,Pani2015Robust,Taira2018} and are robust to radiometric changes that we encounter in long-term localization~\citep{Uy2018}. Semantic representation also offer promising results. Description at object-level is generic and compact in regard to the raw data~\citep{Salas-Moreno2013} and can be used to handle local changes in scene appearance and geometry~\citep{Weinzaepfel2019}. Images, though, contain extremely explanatory clues~\citep{Arandjelovic2017} and are much more easier to collect compared to semantic and geometric data. That is why, when considered as complementary information, these three type of data offer the most effective scene representation for \ac{vbl}~\citep{Schonberger2017a,Taira2019}.
     
		\paragraph{Machine learning in \ac{vbl}.}
			\ac{vbl} benefits from the recent progress in machine learning. Recent global image descriptors for localization are \ac{cnn} especially trained for this task~\citep{Radenovic2017,Gordo2017,Noh2017} (see next chapter, section~\ref{sec:cbir_data_for_loc}). End-to-end \ac{cnn} for 6-\ac{dof} pose regression are also a growing research subject~\citep{Kendall2015,Kendall2016,Kendall2017,Walch2016a,Saha2018,Brahmbhatt2017a,Radwan2018}. Regarding structured methods, classic gradient-based local features (\eg SIFT) are progressively replaced by their learned counterpart~\citep{Sarlin2018a,Rocco2018}. \ac{dl} plays also an important role in the semantically-guided localization methods (section~\ref{subsec:semantic_info}). High performances achieved by recent dense semantic segmentation models have permitted the emergence of novel \ac{vbl} approaches~\citep{Toft2017,Toft2018,Naseer2017a,Shi2019,Schonberger2017a}. \ac{vbl} based on geometric information are also beneficing from \ac{dl} progress. For instance, PointNet~\citep{Qi2016a} has been successfully used to describe point cloud for large-scale localization~\citep{Uy2018}. New models capable of modality transfer, like depth from monocular images \ac{cnn}~\citep{Eigen2014}, are well designed to handle cross-data localization scenario. Its can be used in \ac{vbl}~\citep{Taira2019}, as its have been used in autonomous navigation to improve \ac{slam} systems~\citep{Tateno2017,Loo2019}.
			     
     	\paragraph{Runtime consideration}
	     	\label{para:runtime}
	     	Real-time performances and embedded architectures are constraints mainly present in the robotic community. In \ac{vbl}, such criteria are not always taken into account. This can be explained by the fact that recovering the localization of an input query is a one-shoot action; \ie it has to be performed only once compared to tracking systems~\citep{Marchand2016} or SLAM algorithms~\citep{Garcia-Fidalgo2015}. Furthermore, as described in previous section~\ref{sec:vbl_methods}, \ac{vbl} methods are tow-step: an offline and an online step. Computational time is mainly consumed during the offline step that can be computed in advance of the localization.
	     	
	     	Yet, some authors manage to reduce the computational cost of their system~\citep{Shotton2013,Glocker2015,Lynen2015}. For instance works from~\citep{Feng2016a,Cheng2019} introduce a light version of F2P method, using binary local descriptors, and works from~\citep{Weyand2016,Kendall2015,Contreras2017} embed their localization system in a compact CNN architecture loadable on a smart-phone. \citet{Middelberg2014} introduce a multi-scale scene representation for low-cost computation. The pose is firstly estimated according to a local representation of the scene before a global estimation on the full scene, computed on the cloud. \citet{Sarlin2018a} reduce the size of their \ac{cnn} by training it through distillation.
	    