In this final chapter, we summarize the main contributions of this thesis and, in the last section, we propose future research directions regarding the presented solutions.

\section{Summary of the thesis}

Throughout this thesis, we focus on \ac{vbl} in urban environment. We define the boundaries of our research in the first chapter. In chapter~\ref{chap:2}, we review exhaustively \ac{vbl} domain and methods, with a particular attention paid to challenges induced by long-term localization and to the data heterogeneity presents in the localization approaches. We came out with the conclusion that they are a lake of methods taking advantages of asymmetric data, \ie data without the same completeness regarding the modalities available in queries used during online localization and the modalities of the database used as reference map.

In the chapter~\ref{chap:3}, we propose a new trainable global image descriptor for \ac{cbir} for localization. The particularity of our method remains in the fact that our descriptor can be trained using side modality that is not available during the task of localization. We show that spreading out geometric clues within our pure radiometric descriptor improve the performances in challenging long-term localization scenarios.

Chapter~\ref{chap:4} is dedicated to our relocalization pipeline. Our method aims to improve the localization given by our initial localization step. Using geometric reasoning, we refine 6-\ac{dof} pose of the query to localize. We define our method to be lightweight and easily plugged after an existing \ac{cbir} for localization approach. In a same manner as our global image descriptor, the relocalization is aided by the geometric information learned during an offline stage.
