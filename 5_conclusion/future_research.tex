\section{Future Research}

In this section, we enumerate possible improvement regarding the works presented in this thesis.

\paragraph{Towards an unified \acs{vbl} pipeline.} Our two main proposal have been developed in parallel during this thesis, resulting on two independent architectures. A straightforward improvement of this work will be the unification of these two frameworks, in order to get a complete two stage hierarchical localization method~\citep{Sarlin2018a}. This unification will make possible the strict comparison of our localization results with state-of-the-art \ac{vbl} method thanks to the new challenging localization benchmark of \citet{Sattler2018}.

\paragraph{Multitask-training.} It will be interesting to investigate multitask learning in order to address all the computer vision problems involved in \ac{vbl} jointly. Our localization method involve global image description, establishments of dense correspondences between images and depth map generation from monocular image. We only target one specific training task for our global image descriptor and our refinement method. Optimizing jointly the different tasks involved in our localization pipeline will certainly improve the overall precision of the system.

\paragraph{Heterogeneity in the geometry of acquisition.} In this research work, we propose methods to deal with heterogeneity within the data modality. Indeed, we find a solution to benefit from extra modalities present in the reference data but not in the query side. Another interesting research oriented question will be: how to deal with visual data with different acquisition geometry? To be more specific, it will be interesting to tackle the problem of comparing perspective images with spherical ones~\citep{Iscen2017,Ramalingam2010,Torii2011,Zamir2010,Zamir2014}. Indeed, database coverage can be easily extended by using wide angle or omnidirectional cameras (\eg google street view panorama). Furthermore, recent work have introduced a specific tool to exploit spherical geometry with deep learning : spherical \ac{cnn}~\citep{Cohen2018}. This new architecture has already been successfully used to sole a wide range of problems: room layout recovery from 360 images~\citep{Fernandez-Labrador2019}, depth estimation from spherical panorama~\citep{Zioulis2018}, etc. We are convinced that similar approaches can be used to solve geometrically heterogeneous \ac{vbl} problems.